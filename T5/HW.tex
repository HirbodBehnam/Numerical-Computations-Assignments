% !TEX program = xelatex
\documentclass[]{article}
\usepackage{commons/course}

\begin{document}
\printheader

\section*{سوال اول}
% https://www.math-cs.gordon.edu/courses/ma342/handouts/rate.pdf
\subsection*{قسمت اول}
بسط تیلور
$g(x)$
را تا مرتبه 2 می‌نویسیم. فرض بکنید که
$x^*$
همان نقطه ثابت تابع است. همچنین فعلا فرض بکنید که
$g''(x^*) \neq 0$
است.
\begin{align*}
    g(x) &= g(x^*) + \cancel{\frac{g'(x^*)}{1!} (x - x^*)} + \frac{g''(x^*)}{2!} (x - x^*)^2 + \frac{g'''(\xi)}{3!} (x - x^*)^3\\
    &= x^* + \frac{g''(x^*)}{2} (x - x^*)^2 + \frac{g'''(\xi)}{6} (x - x^*)^3
\end{align*}
حال با کمی جایگزای و تقسیم کردن به عبارت زیر می‌توانیم برسیم:
\begin{gather*}
    \frac{g(x) - x^*}{(x - x^*)^2} = \frac{g''(x^*)}{2} + \frac{g'''(\xi)}{6} (x - x^*)
    \implies
    \lim_{x \rightarrow \infty} \frac{|x_{n+1} - x^*|}{|x_{n} - x^*|^2} = |\frac{g''(x^*)}{2}|
\end{gather*}
پس در نتیجه از آنجا که در کسر توان دو ظاهر شده است پس همگرایی مرتبه دو دارد.
حال فرض بکنید که اگر
$g''(x^*) = 0$
بود چه می‌شد؟ در اینجا اتفاقی که می‌افتاد این بود که باید اینقدر بسط تیلور را ادامه می‌دادیم تا به
جایی برسیم که
$g^{(k)}(x^*) \neq 0$
می‌شد. در اینجا اگر کل بسط را تقسیم بر
$(x - x^*)^k$
می‌کردیم عملا به عبارتی شبیه
$\frac{g(x) - x^*}{(x - x^*)^k}$
می‌رسیدیم که عملا نشان می‌داد که درجه همگرایی برابر $k$ بود.
\subsection*{قسمت دوم}
در ابتدا نقطه‌ی ثابت این دنباله را پیدا می‌کنیم:
\begin{align*}
    x_{0} = f(0) &= 0.25\\
    x_{1} = f(0.25) &= 0.2916666666666667\\
    x_{2} = f(0.2916666666666667) &= 0.2928921568627451\\
    x_{3} = f(0.2928921568627451) &= 0.29289321881265507
\end{align*}
حال داریم
$f'(x) = \frac{4 x}{4 x -4}-\frac{4 \left(2 x^{2}-1\right)}{\left(4 x -4\right)^{2}}$.
پس مشخص است که
$f'(x_3) = 0$
است و در نتیجه می‌توان گفت که از آنجا که این دنبال عملا شرط قسمت الف را دارد پس درجه همگرایی حداقل 2 است.
\section*{سوال دوم}
برای نقطه ثابت کافی است که از نقطه‌ای مثل
$p_0 = 0$
شروع بکنیم و با مثلا
$a = 0.5$
دنباله‌ی ذکر شده را حساب بکنیم. دقت بکنید که مقدار $a$
مهم است چرا که باید مشتق تابع
$g(x) = x - a \times f(x)$
همیشه بین 0 و 1 باشد.
داریم:
\begin{align*}
    p_{0} &= 0\\
    p_{1} = 0 - 0.5 \times f(0) &= 0.25\\
    p_{2} = 0.25 - 0.5 \times f(0.25) &= 0.3450480203727385\\
    p_{3} = 0.3450480203727385 - 0.5 \times f(0.3450480203727385) &= 0.3663980183652408\\
    p_{4} = 0.3663980183652408 - 0.5 \times f(0.3663980183652408) &= 0.3701468307320728\\
    p_{5} = 0.3701468307320728 - 0.5 \times f(0.3701468307320728) &= 0.3707663313950361\\
    p_{6} = 0.3707663313950361 - 0.5 \times f(0.3707663313950361) &= \boxed{0.3708675964326705}
\end{align*}
حال برای نیوتن رافسون در ابتدا باید که
$f'(x)$
را حساب بکنیم. در این سوال داریم که
$f'(x) = 2x + \cos (x)$. حال داریم:
\begin{align*}
    p_{0} &= 0\\
    p_{1} = 0 - \frac{f(0)}{f'(0)} &= 0.5\\
    p_{2} = 0.5 - \frac{f(0.5)}{f'(0.5)} &= 0.37780801587057\\
    p_{3} = 0.37780801587057 - \frac{f(0.37780801587057)}{f'(0.37780801587057)} &= 0.3709105514033993\\
    p_{4} = 0.3709105514033993 - \frac{f(0.3709105514033993)}{f'(0.3709105514033993)} &= \boxed{0.37088734037553595}
\end{align*}
همان طور که مشاهده می‌کنید جواب‌های بدست آمده بسیار نزدیک به هم هستند و اختلاف خیلی کمی دارند و هر دو
دنباله همگرا هستند. حال مرتبه همگرایی هر دو را حساب می‌کنیم. در ابتدا برای روش نقطه ثابت
$f'(0.3708675964326705)$
را حساب می‌کنیم که برابر
$1.67374845136$
است. پر در نتیجه مرتبه همگرایی ۱ است. همچنین مشخص است که این عدد ریشه ساده معادله است. حال
$f''(x)$
را حساب می‌کنیم که در اینجا برابر است با
$2 - \sin (x)$.
مشخص است که
$f''(0.3708675964326705)$
برابر ۰ نیست. پس در نتیجه می‌توان گفت که مرتبه همگرایی حداقل درجه دو است.
\section*{سوال سوم}
برای روش وتری از دو نقطه‌ی ۲ و ۳ که در صورت سوال نیز آمده است شروع می‌کیم و تا جایی پیش می‌رویم که
$|f(p_i)| \le 0.01$
بشود. پس داریم:
\begin{align*}
    p_0 &= 2\\
    p_1 &= 3\\
    p_{2} = 3 - \frac{f(3) (3 - 2)}{f(3) - f(2)} &= 2.6956521739130435\\
    p_{3} = 2.6956521739130435 - \frac{f(2.6956521739130435) (2.6956521739130435 - 3)}{f(2.6956521739130435) - f(3)} &= 2.7182530624667125\\
    p_{4} = 2.7182530624667125 - \frac{f(2.7182530624667125) (2.7182530624667125 - 2.6956521739130435)}{f(2.7182530624667125) - f(2.6956521739130435)} &= \boxed{2.718856457258174}
\end{align*}
از طرفی دیگر برای روش دو بخشی داریم:
\[
\begin{array}{ccc}
    a = 2 & b = 3 & p = \frac{2 + 3}{2} = 2.5\\
    a = 2.5 & b = 3 & p = \frac{2.5 + 3}{2} = 2.75\\
    a = 2.5 & b = 2.75 & p = \frac{2.5 + 2.75}{2} = 2.625\\
    a = 2.625 & b = 2.75 & p = \frac{2.625 + 2.75}{2} = 2.6875\\
    a = 2.6875 & b = 2.75 & p = \frac{2.6875 + 2.75}{2} = \boxed{2.71875}
\end{array}
\]
همان طور که مشخص است و انتظار می‌رفت روش وتری زودتر همگرا می‌شود.
\section*{سوال چهارم}
در ابتدا تابع $f$
را به صورت زیر تعریف می‌کنیم:
\begin{gather*}
    f(x) = 10^x - 56
\end{gather*}
ریشه‌ی این تابع عملا جواب
$\log_{10}(56)$
است. پس کافی است که ریشه‌ی
$f$
را به کمک روش‌های خواسته شده پیدا بکنیم. همچنین از آنجا که بدیهتا جواب بین 1 و 2 است به نظر من شروع از
$p_0 = 1$
مناسب است. همچنین داریم:
\begin{gather*}
    f'(x) = 10^x \times \ln(10) = 2.302585093 \times 10^x
\end{gather*}
پس برای روش نیوتون رافسون داریم:
\begin{align*}
    p_{0} &= 1\\
    p_{1} = 1 - \frac{f(1)}{f'(1)} &= 2.997754616754958\\
    p_{2} = 2.997754616754958 - \frac{f(2.997754616754958)}{f'(2.997754616754958)} &= 2.5879066929143026\\
    p_{3} = 2.5879066929143026 - \frac{f(2.5879066929143026)}{f'(2.5879066929143026)} &= 2.2164275409696246\\
    p_{4} = 2.2164275409696246 - \frac{f(2.2164275409696246)}{f'(2.2164275409696246)} &= 1.9298889474120062\\
    p_{5} = 1.9298889474120062 - \frac{f(1.9298889474120062)}{f'(1.9298889474120062)} &= 1.7814083950339017\\
    p_{6} = 1.7814083950339017 - \frac{f(1.7814083950339017)}{f'(1.7814083950339017)} &= 1.749426799236975\\
    p_{7} = 1.749426799236975 - \frac{f(1.749426799236975)}{f'(1.749426799236975)} &= \boxed{1.7481897920512308}
\end{align*}
همچنین برای روش نقطه ثابت داریم:
(دقت کنید که از آنجا که رشد تابع نمایی بسیار زیاد است پس $a$ را باید کم بگیریم. من در اینجا $a=0.01$ قرار دادم.)
\begin{align*}
    a_{0} &= 1\\
    a_{1} = 1 - 0.01 \times f(1) &= 1.46\\
    a_{2} = 1.46 - 0.01 \times f(1.46) &= 1.7315968496873393\\
    a_{3} = 1.7315968496873393 - 0.01 \times f(1.7315968496873393) &= 1.7525868159738938\\
    a_{4} = 1.7525868159738938 - 0.01 \times f(1.7525868159738938) &= 1.7468859859305437\\
    a_{5} = 1.7468859859305437 - 0.01 \times f(1.7468859859305437) &= 1.7485623855092107\\
    a_{6} = 1.7485623855092107 - 0.01 \times f(1.7485623855092107) &= 1.7480794617080704\\
    a_{7} = 1.7480794617080704 - 0.01 \times f(1.7480794617080704) &= \boxed{1.7482194334809817}
\end{align*}
\section*{سوال پنجم}
% https://math.stackexchange.com/a/721497/424863
همان طور که در صورت سوال ذکر شده است
$0 < a < f'(x) < b$
است. این نشان می‌دهد که یک عدد مثل
$M$
وجود دارد که
$0 < Mf'(x) < 1$
باشد. و به صورت کلی‌تر اگر داشته باشیم
$g(x) = x - Mf(x)$
آنگاه
$0 < g'(x) < 1$.
حال فرض بکنید که نقطه ثابت تابع
$g$
را با
$x^*$
نشان می‌دهیم. آنگاه اول از همه مشخص است که باید
$f(x^*) = 0$
باشد که
$g(x^*) = x^*$
بشود. پس در صورتی که
$x^*$
را پیدا بکنیم جواب
$f(x^*) = 0$
را پیدا کرده‌ایم. حال فرض بکنید که 
می‌خواهیم در هر مرحله که
$x_{n+1}$
را حساب می‌کنیم،
$|x_{n+1} - x^*|$
را حساب بکنیم. دقت بکنید که این عبارت عملا همان
$|g(x_n) - g(x^*)|$
است. حال نکته‌ای که باید توجه بکنید این است که از آنجا که
$|g'(x)| < 1$
است پس طبق تعریف مشتق:
\begin{gather*}
    |g(x_n) - g(x^*)| < |x_n - x^*| \implies |x_{n+1} - x^*| < |x_n - x^*|
\end{gather*}
این نشان می‌دهد که هر بار که جلو می‌رویم
$x_n$ به $x^*$
نزدیک‌تر می‌شود و در بی‌نهایت به آن میل می‌کند.
\end{document}

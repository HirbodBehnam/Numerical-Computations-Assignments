% !TEX program = xelatex
\documentclass[]{article}
\usepackage{commons/course}

\begin{document}
\printheader

\section*{سوال اول}
عملا پس نقطه‌هایی که باید
interpolation
را روی آن‌ها بزنیم برابر هستند با
$\left(x_i, \left(x_i\right)^n\right)$
هستند. پس عملا
interpolation
ما برابر است با
\begin{gather*}
    \sum_{k=0}^{n} (x_k)^n L_{n, k} (x) = \sum_{k=0}^{n} \Bigl((x_k)^n \prod_{i=0, i \neq k}^{n} \frac{x - x_i}{x_k - x_i}\Bigr)
\end{gather*}
حال ضریب‌های خواسته شده را بررسی می‌کنیم. در ابتدا مشخص است که برای
$x^n$
در داخل
$\prod$
هیچ ضریبی برای
$x$ها
در صورت وجود ندارد و در نتیجه پس ضریب
$x^n$
برابر است با:
\begin{gather*}
    \sum_{k=0}^{n} \Bigl( (x_k)^n \prod_{i=0, i \neq k}^{n} \frac{1}{x_k - x_i} \Bigr)
\end{gather*}
حال ضریب جمله‌ی
$x^{n-1}$
را حساب می‌کنیم. می‌دانیم که ضریب جلمه‌ی
$x^{n-1}$
در
$\prod_{i=1}^n (x + x_i)$
برابر
$\sum_{i=1}^n x_i$
است. پس در نتیجه ضریب جمله
$x^{n-1}$
در عبارت
interpolation
برابر است با
\begin{gather*}
    \sum_{k=0}^{n} \Bigl((x_k)^n \bigl( \sum_{i=0, i \neq k}^{n} -x_i  \bigr) \bigl( \prod_{i=0, i \neq k}^{n} \frac{1}{x_k - x_i} \bigr) \Bigr)
\end{gather*}
یک نکته که باید توجه بکنید این است که از آنجا که ما
$n+1$
نقطه داریم و تابع‌ای که در حال تخمین زدن آن هستیم یک چند جمله‌ای است، پس ضریب
$x^n$
برابر 1 است و ضریب
$x^{n-1}$
برابر 0 است. به عبارتی به صورت کلی باید
\lr{interpolation}
ما کلا برابر
$x^n$
شود و هیچ جمله‌ی دیگری نداشته باشد.
\subsection*{قسمت اول}
حال فرض کنید که نقاط ما برابر
$(i, i^n)$
هستند که
$0 \le i \le n$
است. در این حالت داریم برای ضریب جمله‌های داده شده که:
\begin{align*}
    \sum_{k=0}^{n} \Bigl( k^n \prod_{i=0, i \neq k}^{n} \frac{1}{k - i} \Bigr)
    &= \sum_{k=0}^{n} \Bigl( k^n \bigl((-1)^{n-k} \frac{1}{k!} \frac{1}{(n - k)!} \bigr) \Bigr)\\
    &= \boxed{\sum_{k=0}^{n} \Bigl(k^n \frac{(-1)^{n-k}}{k!(n - k)!}\Bigr) = 1}\\
    \sum_{k=0}^{n} \Bigl(k^n \bigl( \sum_{i=0, i \neq k}^{n} -i  \bigr) \bigl(\prod_{i=0, i \neq k}^{n} \frac{1}{k - i} \bigr) \Bigr)
    &= \sum_{k=0}^{n} \Bigl(k^n (k - \frac{n(n+1)}{2}) \bigl((-1)^{n-k} \frac{1}{k!} \frac{1}{(n - k)!} \bigr) \Bigr)\\
    &= \sum_{k=0}^{n} \Bigl(k^{n+1} \frac{(-1)^{n-k}}{k!(n - k)!} \Bigr) - \frac{n(n+1)}{2} \stackrel{1}{\cancel{\sum_{k=0}^{n} \Bigl(k^n \frac{(-1)^{n-k}}{k!(n - k)!}\Bigr)}}\\
    &= 0\\
    &\implies\\
    \sum_{k=0}^{n} \Bigl(k^{n+1} \frac{(-1)^{n-k}}{k!(n - k)!} \Bigr) &= \frac{n(n+1)}{2}\\
    &\stackrel{\times n!}{\implies}\\
    \sum_{k=0}^{n} \Bigl((-1)^{n-k} k^{n+1} \frac{n!}{k!(n - k)!} \Bigr) &= \frac{n(n+1)!}{2}
\end{align*}
\begin{gather*}
    \boxed{\sum_{k=0}^{n} (-1)^{n-k} {n \choose k} k^{n+1} = \frac{n(n+1)!}{2}}
\end{gather*}
\subsection*{قسمت دوم}
حال این بار فرض بکنید که نقاط ما برابر
$(a - i,(a - i)^n)$
هستند که
$a$
یک عدد دلخواه هست و
$0 \le i \le n$
است. دوباره مثل قسمت قبل داریم:
\begin{align*}
    \sum_{k=0}^{n} \Bigl( (a-k)^n \prod_{i=0, i \neq k}^{n} \frac{1}{(a-k) - (a-i)} \Bigr)
    &= \sum_{k=0}^{n} \Bigl( (a-k)^n \prod_{i=0, i \neq k}^{n} \frac{1}{i - k} \Bigr)\\
    &= \sum_{k=0}^{n} \Bigl( (a-k)^n \bigl((-1)^{n-k} \frac{1}{k!} \frac{1}{(n - k)!} \bigr) \Bigr)\\
    &= \sum_{k=0}^{n} \Bigl((a-k)^n \frac{(-1)^{n-k}}{k!(n - k)!}\Bigr) = 1\\
    &\stackrel{\times n!}{\implies}\\
    n! &= \sum_{k=0}^{n} \Bigl((a-k)^n (-1)^{n-k} \frac{n!}{k!(n - k)!}\Bigr)
\end{align*}
\begin{gather*}
    \boxed{\sum_{k=0}^{n} (-1)^{n-k} {n \choose k} (a-k)^n = n!}
\end{gather*}
\section*{سوال دوم}
در ابتدا می‌‌دانیم که
$\Delta^2P(0)$
عملا برابر است با
$\Delta P(1) - \Delta P(0)$.
فرض کنید که
$P(x) = ax^4 + bx^3 + cx^2 + dx + e$
است.
پس داریم:
\begin{align*}
    \Delta^4P(0) &= \Delta^3 P(1) - \Delta^3 P(0)\\
    &= \Delta^2 P(2) - \Delta^2 P(1) - (\Delta^2 P(1) - \Delta^2 P(0))\\
    &= \Delta^2 P(2) - 2\Delta^2 P(1) + \Delta^2 P(0)\\
    &= \Delta P(3) - \Delta P(2) - 2(\Delta P(2) - \Delta P(1)) + \Delta P(1) - \Delta P(0)\\
    &= \Delta P(3) - 3\Delta P(2) + 3\Delta P(1) - \Delta P(0)\\
    &= P(4) - P(3) - 3(P(3) - P(2)) + 3(P(2) - P(1)) - (P(1) - P(0))\\
    &= P(4) - 4P(3) + 6P(2) - 4P(1) + P(0)\\
    &= 24a = 24\\
    &\implies \boxed{a = 1}
\end{align*}
همچنین برای
$\Delta^3 P(0)$
داریم:
\begin{align*}
    \Delta^3P(0) &= \Delta^2 P(1) - \Delta^2 P(0)\\
    &= \Delta P(2) - \Delta P(1) - (\Delta P(1) - \Delta P(0))\\
    &= \Delta P(2) - 2\Delta P(1) + \Delta P(0)\\
    &= P(3) - P(2) - 2(P(2) - P(1)) + P(1) - P(0)\\
    &= P(3) - 3P(2) + 3P(1) - P(0)\\
    &= 36a + 6b = 6\\
    &\implies \boxed{b = -5}
\end{align*}
همچنین برای
$\Delta^2 P(0)$
داریم:
\begin{align*}
    \Delta^2P(0) &= \Delta P(1) - \Delta P(0)\\
    &= P(2) - P(1) - (P(1) - P(0))\\
    &= P(2) - 2 P(1) + P(0)\\
    &= 14a + 6b + 2c = 0\\
    &\implies \boxed{c = 8}
\end{align*}
در نهایت برای خود خواسته‌ی سوال داریم:
\begin{align*}
    \Delta^2P(10) &= \Delta P(11) - \Delta P(10)\\
    &= P(12) - P(11) - (P(11) - P(10))\\
    &= P(12) - 2 P(11) + P(10)\\
    &= 1454a + 66b + 2c = 0\\
    &\implies \boxed{\Delta^2P(10) = 1140}
\end{align*}
\section*{سوال سوم}
اسپلاین مرتبه‌ی اول در اصل همان
piecewise
است. پس کافی است که بین هر دو نقطه‌ای که داریم یک خط بکشیم. فرض کنید که نقطه‌ها به صورت
$(x_i, y_i), (x_{i+1}, y_{i+1})$
است. در این حالت معادله‌ی خط برابر است با:
\begin{align*}
    S(x) &= \frac{y_{i+1} - y_i}{x_{i+1} - x_i}x + y_i - \frac{y_{i+1} - y_i}{x_{i+1} - x_i}x_i\\
    &=\frac{y_{i+1} - y_i}{x_{i+1} - x_i}(x-x_i) + y_i
\end{align*}
که عملا اگر فکر بکنیم متوجه می‌شویم که این عملا همان درون یابی نیوتن است که بین دو نقطه
اتفاق افتاده است. عملا در اینجا از آنجا که این درون یابی جواب درون یابی لاگرانژ نیز است می‌توان
از فرمول باقی‌مانده لاگرانژ از
\link{https://en.wikipedia.org/wiki/Polynomial_interpolation\#Interpolation_error:_Lagrange_remainder_formula}{ویکیپدیا}
استفاده کرد.
\begin{align*}
    |f(x)-S(x)|&=|\frac {f''(\xi )}{2!}\prod _{i=0}^{2}(x-x_{i})|\\
    &=|\frac{f''(\xi)}{2}(x-x_{i})(x-x_{i+1})|
\end{align*}
از آنجایی که اختلاف هر
$x_i$ و $x_{i+1}$
یکی است پس در تمامی بازه‌ها مقدار حداکثر
$|(x-x_{i})(x-x_{i+1})|$
ثابت است. پس صرفا باید چک کنیم که در بازه‌های مختلف مقدار
$|f''(\xi)|$
چه تغییری می‌کند. با توجه به صورت سوال نیز از آنجا که مشتق دوم عبارت نزولی است پس همیشه
$|f''(\xi)|$
در حال کاهش است. پس در نتیجه هر چه قدر که جلوتر می‌رویم در
\lr{spline}
خطا کمتر می‌شود. پس بیشترین خطای
\lr{spline}
در همان تیکه‌ی اول است و حکم مسئله ثابت شد.
\section*{سوال چهارم}
\subsection*{قسمت الف}
طبق کتاب داریم:
\[
\begin{bmatrix}
    1 & 0 & 0 & 0 \\
    h_0 & 2(h_0+h_1) & h_1 & 0 \\
    0 & h_1 & 2(h_1+h_2) & h_2\\
    0 & 0 & 0 & 1 \\
\end{bmatrix}
\begin{bmatrix}
    c_0\\
    c_1\\
    c_2\\
    c_3
\end{bmatrix}
=
\begin{bmatrix}
    0\\
    \frac{3}{h_1}(a_2 - a_1) - \frac{3}{h_0}(a_1 - a_0)\\
    \frac{3}{h_2}(a_3 - a_2) - \frac{3}{h_1}(a_2 - a_1)\\
    0
\end{bmatrix}
\]
همچنین پارامتر‌های ما به صورت زیر هستند:
\begin{gather*}
    h_0 = 0 - (-1) = 1, \quad h_1 = 1 - 0 = 1, \quad h_2 = 2 - 1 = 1\\
    f(x_i) = a_i
\end{gather*}
پس معادله‌ی ما به صورت زیر است:
\[
\begin{bmatrix}
    1 & 0 & 0 & 0 \\
    1 & 4 & 1 & 0 \\
    0 & 1 & 4 & 1\\
    0 & 0 & 0 & 1 \\
\end{bmatrix}
\begin{bmatrix}
    c_0\\
    c_1\\
    c_2\\
    c_3
\end{bmatrix}
=
\begin{bmatrix}
    0\\
    1.5\\
    3\\
    0
\end{bmatrix}
\]
پس داریم:
\begin{gather*}
    c_0 = 0\\
    c_1 = 0.2\\
    c_2 = 0.7\\
    c_3 = 0
\end{gather*}
\[
\begin{array}{cccc}
    a_0 = 0.5 & b_0 = (a_1 - a_0)/h_0 - h_0 (c_1-c_0) = 0.43333333 & c_0 = 0 & d_0 = (c_1 - c_0) / (3h_0) = 0.06666667\\
    a_1 = 1 & b_1 = (a_2 - a_1)/h_1 - h_1 (c_2-c_1) = 0.63333333 & c_1 = 0.2 & d_1 = (c_2 - c_1) / (3h_1) = 0.16666667\\
    a_2 = 2 & b_2 = (a_3 - a_2)/h_2 - h_2 (c_3-c_2) = 1.53333333 & c_2 = 0.7 & d_2 = (c_3 - c_2) / (3h_2) = -0.23333333
\end{array}
\]
پس معادلات به صورت زیر هستند:
\begin{gather*}
    0.5 + 0.43333333 (x + 1) + 0.06666667 (x + 1)^3 \quad -1 \le x \le 0\\
    1 + 0.63333333 x + 0.2x^2 + 0.16666667 x^3 \quad 0 \le x \le 1\\
    2 + 1.53333333 (x - 1) + 0.7 (x-1)^2 -0.23333333 (x - 1)^3 \quad 1 \le x \le 2
\end{gather*}
\subsection*{قسمت ب}
طبق کتاب داریم:
\[
\begin{bmatrix}
    2h_0 & h_0 & 0 & 0 \\
    h_0 & 2(h_0+h_1) & h_1 & 0 \\
    0 & h_1 & 2(h_1+h_2) & h_2\\
    0 & 0 & h_2 & 2h_2 \\
\end{bmatrix}
\begin{bmatrix}
    c_0\\
    c_1\\
    c_2\\
    c_3
\end{bmatrix}
=
\begin{bmatrix}
    \frac{3}{h_0} (a_1 - a_0) - 3f'(a)\\
    \frac{3}{h_1}(a_2 - a_1) - \frac{3}{h_0}(a_1 - a_0)\\
    \frac{3}{h_2}(a_3 - a_2) - \frac{3}{h_1}(a_2 - a_1)\\
    3f'(b) - \frac{3}{h_{n-1}} (a_n - a_{n-1})
\end{bmatrix}
\]
همچنین پارامتر‌های ما به صورت زیر هستند:
\begin{gather*}
    h_0 = 0 - (-1) = 1, \quad h_1 = 1 - 0 = 1, \quad h_2 = 2 - 1 = 1\\
    f(x_i) = a_i
\end{gather*}
پس معادله‌ی ما به صورت زیر است:
\[
\begin{bmatrix}
    2 & 1 & 0 & 0 \\
    1 & 4 & 1 & 0 \\
    0 & 1 & 4 & 1\\
    0 & 0 & 1 & 2 \\
\end{bmatrix}
\begin{bmatrix}
    c_0\\
    c_1\\
    c_2\\
    c_3
\end{bmatrix}
=
\begin{bmatrix}
    -1.5\\
    1.5\\
    3\\
    -9
\end{bmatrix}
\]
پس داریم:
\begin{gather*}
    c_0 = -0.76666667\\
    c_1 = 0.03333333\\
    c_2 = 2.13333333\\
    c_3 = -5.56666667
\end{gather*}
\[
\begin{array}{cccc}
    a_0 = 0.5 & b_0 = 1 & c_0 = -0.76666667 & d_0 = 0.26666667\\
    a_1 = 1 & b_1 = 0.26666667 & c_1 = 0.03333333 & d_1 = 0.7\\
    a_2 = 2 & b_2 = 2.43333333 & c_2 = 2.13333333 & d_2 = -2.56666667
\end{array}
\]
پس معادلات به صورت زیر هستند:
\begin{gather*}
    0.5 + (x + 1) -0.76666667 (x + 1)^2 + 0.26666667 (x + 1)^3 \quad -1 \le x \le 0\\
    1 + 0.26666667 x + 0.03333333x^2 + 0.7 x^3 \quad 0 \le x \le 1\\
    2 + 2.43333333 (x - 1) + 2.13333333 (x-1)^2 -2.56666667 (x - 1)^3 \quad 1 \le x \le 2
\end{gather*}
\section*{سوال پنجم}
به صورت کلی هیچ مشکلی پیش نمی‌آید اگر دو شرط زیر را نیز به شرط‌های چند جمله‌ای اضافه بکنیم.
\begin{align*}
    f'(0) &= P'(0)\\
    f'(2) &= P'(2)
\end{align*}
با این کار از
\lr{Hermite interpolation}
استفاده می‌کنیم که چند جمله‌ای را تخمین بزنیم.
حال مقدار‌های چند جمله‌ای در نقاط خواسته شده را حساب می‌کنیم.
\begin{gather*}
    P(0) = 0, P(1) \approx 0.6931, P(2) = 1.0986\\
    P'(0) = 1, P'(1) \approx \frac{1}{2}, P'(2) = \frac{1}{3}
\end{gather*}
پس داریم:
\[
\begin{array}{ccccccc}
z_0 = 0 & f[z_0] = 0 \\
&& f'(0) = 1 \\
z_1 = 0 & f[z_1] = 0 && f[z_0,z_1,z_2] = -0.3069\\
&& f[z_1,z_2] = 0.6931 && 0.1138\\
z_2 = 1 & f[z_2] = 0.6931 && f[z_1,z_2,z_3] = -0.1931 && -0.0323\\
&& f'(1) = \frac{1}{2} && 0.0493 && 0.0092\\
z_3 = 1 & f[z_3] = 0.6931 && f[z_2,z_3,z_4] = -0.0945 && -0.014 \\
&& f[z_3,z_4] = 0.4055 && 0.0223 \\
z_4 = 2 & f[z_4] = 1.0986&&f[z_3,z_4,z_5] = -0.0722\\
&&f'(2) = \frac{1}{3}\\
z_5 = 2 & f[z_5] = 1.0986\\
\end{array}
\]
پس چند جمله‌ای ما به صورت زیر است:
\begin{gather*}
    0 + x -0.3069 x (x-1) + 0.1138 x (x-1)^2 -0.0323 x (x-1)^2 (x-2) + 0.0092 x (x-1)^2 (x-2)^2
\end{gather*}
حال برای خطا داریم:
\begin{gather*}
    \frac{f^{(3)}(\xi)}{3!}\prod_i(x - x_i)^{k_i}, \quad 0 \le \xi \le 2
\end{gather*}
از آنجا که مشتق سوم نزولی است پس زمانی که
$\xi = 0$
باشد بیشترین مقدار 2 را دارد مشتق. پس داریم:
\begin{gather*}
    \frac{2}{6} (x - 0)^2 (x - 1)^2 (x - 2)^2 
\end{gather*}
که
$x$
در اینجا نقطه‌‌ای است که می‌خواهیم مقدار آن را تخمین بزنیم.
\section*{سوال ششم}
\link{https://math.stackexchange.com/a/159883/424863}{منبع}

\noindent
فرض کنید که
$g(x) = f[x, x_0, x_1, \dots, x_n]$
است. حال
$g[u, v]$
را در نظر بگیرید. مشخص است که طبق تعریف که
$g[u, v] = \frac{g(u) - g(v)}{u - v}$
است مشخص است که
$\lim_{(u, v) \rightarrow (x, x)} g[u, v] = \frac{d}{dx} g(x)$.
همچنین می‌دانیم که
$\frac{g(u) - g(v)}{u - v}$
همان
$\frac{f[u, x_0, x_1, \dots, x_n] - f[v, x_0, x_1, \dots, x_n]}{u - v}$
است که در اینجا برابر می‌شود با
$f[x, x, x_0, x_1, \dots, x_n]$.
\section*{سوال هفتم}
به کمک درونیابی لاگرانژ داریم:
\begin{gather*}
    p_2^{(0,2)} = y_0 \frac{(x-x_1)(x-x_2)}{(x_0-x_1)(x_0-x_2)} + y_1 \frac{(x-x_0)(x-x_2)}{(x_1-x_0)(x_1-x_2)} + y_2 \frac{(x-x_0)(x-x_1)}{(x_2-x_0)(x_2-x_1)}\\
    p_2^{(1,3)} = y_1 \frac{(x-x_2)(x-x_3)}{(x_1-x_2)(x_1-x_3)} + y_2 \frac{(x-x_1)(x-x_3)}{(x_2-x_1)(x_2-x_3)} + y_3 \frac{(x-x_1)(x-x_2)}{(x_3-x_1)(x_3-x_2)}
\end{gather*}
همچنین مشخص است که
\begin{align*}
    \frac{(x-x_0)p_2^{(1,3)} - (x-x_3)p_2^{(0,2)}}{x_3-x_0} &=
    y_1 \frac{(x-x_0)(x-x_2)(x-x_3)}{(x_1-x_2)(x_1-x_3)(x_3-x_0)} + y_2 \frac{(x-x_0)(x-x_1)(x-x_3)}{(x_2-x_1)(x_2-x_3)(x_3-x_0)}\\
    &\quad + y_3 \frac{(x-x_0)(x-x_1)(x-x_2)}{(x_3-x_1)(x_3-x_2)(x_3-x_0)}\\
    &\quad - y_0 \frac{(x-x_1)(x-x_2)(x-x_3)}{(x_0-x_1)(x_0-x_2)(x_3-x_0)} - y_1 \frac{(x-x_0)(x-x_2)(x-x_3)}{(x_1-x_0)(x_1-x_2)(x_3-x_0)}\\
    &\quad - y_2 \frac{(x-x_0)(x-x_1)(x-x_3)}{(x_2-x_0)(x_2-x_1)(x_3-x_0)}
\end{align*}
با کمی مرتب سازی داریم:
\begin{align*}
    \frac{(x-x_0)p_2^{(1,3)} - (x-x_3)p_2^{(0,2)}}{x_3-x_0} &=
    y_0 \frac{(x-x_1)(x-x_2)(x-x_3)}{(x_0-x_1)(x_0-x_2)(x_0-x_3)}\\
    &\quad + y_1 \frac{(x-x_0)(x-x_2)(x-x_3)(\cancel{x_1}-x_0)-(x-x_0)(x-x_2)(x-x_3)(\cancel{x_1}-x_3)}{(x_1-x_2)(x_1-x_3)(x_3-x_0)(x_1-x_0)}\\
    &\quad + y_2 \frac{(x-x_0)(x-x_1)(x-x_3)(\cancel{x_2}-x_0)-(x-x_0)(x-x_1)(x-x_3)(\cancel{x_2}-x_3)}{(x_2-x_0)(x_2-x_1)(x_3-x_0)}\\
    &\quad + y_3 \frac{(x-x_0)(x-x_1)(x-x_2)}{(x_3-x_1)(x_3-x_2)(x_3-x_0)}\\
    &= y_0 \frac{(x-x_1)(x-x_2)(x-x_3)}{(x_0-x_1)(x_0-x_2)(x_0-x_3)}\\
    &\quad + y_1 \frac{(x-x_0)(x-x_2)(x-x_3)\cancel{(x_3-x_0)}}{(x_1-x_2)(x_1-x_3)\cancel{(x_3-x_0)}(x_1-x_0)}\\
    &\quad + y_2 \frac{(x-x_0)(x-x_1)(x-x_3)\cancel{(x_3-x_0)}}{(x_2-x_0)(x_2-x_1)\cancel{(x_3-x_0)}(x_2-x_3)}\\
    &\quad + y_3 \frac{(x-x_0)(x-x_1)(x-x_2)}{(x_3-x_1)(x_3-x_2)(x_3-x_0)}
\end{align*}
که این همان درون یابی لاگرانژ چهار نقطه‌ی داده شده است.
\section*{سوال هشتم}
\end{document}

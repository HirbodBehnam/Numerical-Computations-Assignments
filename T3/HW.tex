% !TEX program = xelatex
\documentclass[]{article}
\usepackage{commons/course}

\begin{document}
\printheader

\section*{سوال اول}

\section*{سوال دوم}
به صورت کلی از آنجا که
$n = 4$
است. پس نقاط ما برابر نقاط زیر هستند:
\begin{gather*}
    h = \frac{0.5 - 0.1}{4} = 0.1\\
    (0.1, 9.95), (0.2, 4.9), (0.3, 3.18), (0.4, 2.3), (0.5, 1.76)
\end{gather*}
\subsection*{روش ذوزنقه‌ای}
کافی است که از فرمول داخل کتاب استفاده بکنیم:
\begin{align*}
    \int_{0.1}^{0.5} \frac{\cos x}{x} dx &= \frac{h}{2} (f_0 + 2f_1 + 2f_2 + 2f_3 + f_4)\\
    &= \frac{0.1}{2} (9.95 + 2 \times 4.9 + 2 \times 3.18 + 2 \times 2.3 + 1.76)\\
    &= \boxed{1.6235}
\end{align*}
\subsection*{روش سیمپسون}
کافی است که از فرمول داخل کتاب استفاده بکنیم:
\begin{align*}
    \int_{0.1}^{0.5} \frac{\cos x}{x} dx &= \frac{h}{3} (f_0 + 4f_1 + 2f_2 + 4f_3 + f_4)\\
    &= \frac{0.1}{3} (9.95 + 4 \times 4.9 + 2 \times 3.18 + 4 \times 2.3 + 1.76)\\
    &= \boxed{1.56233}
\end{align*}
\section*{سوال سوم}
\subsection*{روش ذوزنقه‌ای}
با توجه به فرمول داریم:
\begin{gather*}
    h = \sqrt{\frac{12 \times 0.001}{(1 - 0) |\max f''(x)|}}
\end{gather*}
حال باید
$|\max \operatorname{cosh}''(x)|$
زمانی که
$0 \le x \le 1$
است را پیدا بکنیم. از آنجا که اگر دو بار از
$\operatorname{cosh}(x)$
مشتق بگیریم همان
$\operatorname{cosh}(x)$
می‌شود پس در نتیجه ماکس آن در
$x = 1$
که نتیجه می‌دهد:
$|\max \operatorname{cosh}''(x)| \approx 1.543$
پس در نتیجه در کل داریم:
\begin{gather*}
    h = \sqrt{\frac{12 \times 0.001}{(1 - 0) 1.543}} \approx 0.088\\
    n > \frac{1 - 0}{0.088} = 11.3636363636\\
    \implies \boxed{n \ge 12}
\end{gather*}
\subsection*{روش سیمپسون}
داریم که خطا برابر است با:
\begin{gather*}
    \frac{b - a}{180}h^4 f^{(4)}(x)
\end{gather*}
در نتیجه مثل قسمت قبل داریم که
\begin{gather*}
    h = \sqrt[4]{\frac{180 \times 0.001}{(1 - 0) |f^{(4)}(x)|}}
\end{gather*}
دوباره به کمک
\lr{desmos}
داریم که حداکثر
$|f^{(4)}(x)|$
برابر با
$1.543$
است در نقطه‌ی 1. پس داریم:
\begin{gather*}
    h = \sqrt[4]{\frac{180 \times 0.001}{(1 - 0) 1.543}} \approx 0.584\\
    n > \frac{1 - 0}{0.584} = 1.71232876712\\
    \implies \boxed{n \ge 2}
\end{gather*}
\section*{سوال چهارم}
\subsection*{قسمت اول}
کافی است که از فرمول‌های سه نقطه‌دار استفاده بکنیم. به عنوان مثال در اینجا از فرمول
\lr{midpoint}
استفاده می‌کنیم.
\begin{align*}
    f'(4.5) &\approx \frac{1}{2h} (f(4.5 + 0.1) - f(4.5 - 0.1))\\
    &=\frac{1}{0.2} (-0.1087 - (-0.1136))\\
    &=\boxed{0.0245}
\end{align*}
\subsection*{قسمت دوم}
کافی است که از فرمول
\lr{forward}
استفاده بکنیم که بسیار شبیه تعریف خود مشتق است:
\begin{align*}
    f'(3) &\approx \frac{f(3 + 0.1) - f(3)}{h}\\
    &=\frac{-0.1612 - (-0.1666)}{0.1}\\
    &=\boxed{0.054}
\end{align*}
\section*{سوال پنجم}
در این قسمت باید از بسط تیلور استفاده بکنیم تا بتوانیم مشتق مرتبه سوم را حساب بکنیم.
\begin{align*}
    f(x_0 + h)  &= f(x_0) + \frac{f'(x_0)h}{1!} + \frac{f''(x_0)h^2}{2!} + \frac{f'''(x_0)h^3}{3!} + O(h^4)\\
    f(x_0 + 2h) &= f(x_0) + \frac{f'(x_0)2h}{1!} + \frac{f''(x_0)4h^2}{2!} + \frac{f'''(x_0)8h^3}{3!} + O(h^4)\\
    f(x_0 + 3h) &= f(x_0) + \frac{f'(x_0)2h}{1!} + \frac{f''(x_0)4h^2}{2!} + \frac{f'''(x_0)8h^3}{3!} + O(h^4)\\
    f(x_0 + 4h) &= f(x_0) + \frac{f'(x_0)2h}{1!} + \frac{f''(x_0)4h^2}{2!} + \frac{f'''(x_0)8h^3}{3!} + O(h^4)\\
    f(x_0 + 5h) &= f(x_0) + \frac{f'(x_0)2h}{1!} + \frac{f''(x_0)4h^2}{2!} + \frac{f'''(x_0)8h^3}{3!} + O(h^4)\\
\end{align*}
\section*{سوال ششم}
\end{document}

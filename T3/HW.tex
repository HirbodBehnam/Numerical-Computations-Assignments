% !TEX program = xelatex
\documentclass[]{article}
\usepackage{commons/course}

\begin{document}
\printheader

\section*{سوال اول}
\subsection*{قسمت اول}
در ابتدا می‌دانیم که
$T(n) = \frac{h}{2}(f(x_0) + 2f(x_1) + \dots + 2f(x_{n-2}) + f(x_{n-1}))$
و
$M(n) = h \sum_{i=0}^{n-1} f(x_i + \frac{h}{2})$
است. در صورتی که این دو عبارت را با هم جمع بکنیم داریم:
\begin{gather*}
    T(n) + M(n) = \frac{h}{2}(f(x_0) + 2f(x_0 + \frac{h}{2}) + 2f(x_1) + \dots + 2f(x_{n-2}) + 2f(x_{n-2} + \frac{h}{2}) + f(x_{n-1}))
\end{gather*}
مشخص است که
$x_{i+1} - x_i = h$
است. پس عملا آن نقاطی که اضافه کردیم دقیقا در وسط هر دو نقطه قدیمی افتند. این بدین معنی
است که انگار تعداد نقاط داخل بازه
$a$ تا $b$
را دو برابر کردیم. همچنین در نهایت نیز باید عبارت را تقسیم بر
$2$
بکنیم که این باعث می‌شود که
$\frac{h}{2}$ تبدیل به $\frac{h}{4}$
شود و عملا همان فرمول ذوزنقه‌ای برای
$2n$
نقطه شود.
\subsection*{قسمت دوم}
اثبات کردن تساوی وسط که صرفا به کمک خطای بسط لاگرانژ است. کافی است که برای هر بخش که تقسیم کرده‌ایم
در نمودار به کمک بسط لاگرانژ یک خط رسم کنیم بین دو نقطه
$x_0$ و $x_0 + h/2 = x_1$
و در نتیجه خطای لاگرانژ برابر می‌شود با:
\begin{gather*}
    \sum_{i=0}^{n-1} \frac{f''(\xi)}{2!} (x - x_i) (x - x_{i+1})
\end{gather*}
پس با انتگرال گیری داریم:
\begin{align*}
    \int_{a}^{b} \sum_{i=0}^{n-1} \frac{f''(\xi)}{2!} (x - x_i) (x - x_{i+1}) dx &=
    \sum_{i=0}^{n-1} \int_{x_i}^{x_{i+1}} \frac{f''(\xi)}{2!} (x - x_i) (x - x_{i+1}) dx\\
    &= \sum_{i=0}^{n-1} \frac{f''(\xi)}{2} \int_{x_i}^{x_{i+1}} x^2 - (x_i + x_i)x + x_i x_{i+1} dx\\
    &= \sum_{i=0}^{n-1} \frac{f''(\xi)}{2} \frac{-(x_{i+1} - x_i)^3}{6}\\
    &= -\sum_{i=0}^{n-1} \frac{f''(\xi)(h/2)^3}{12}\\
    &= -f''(\xi)\frac{(h/2)^3\times n}{12} = \boxed{\frac{b - a}{12} (h/2)^2 f''(\xi)}
\end{align*}
\section*{سوال دوم}
به صورت کلی از آنجا که
$n = 4$
است. پس نقاط ما برابر نقاط زیر هستند:
\begin{gather*}
    h = \frac{0.5 - 0.1}{4} = 0.1\\
    (0.1, 9.95), (0.2, 4.9), (0.3, 3.18), (0.4, 2.3), (0.5, 1.76)
\end{gather*}
\subsection*{روش ذوزنقه‌ای}
کافی است که از فرمول داخل کتاب استفاده بکنیم:
\begin{align*}
    \int_{0.1}^{0.5} \frac{\cos x}{x} dx &= \frac{h}{2} (f_0 + 2f_1 + 2f_2 + 2f_3 + f_4)\\
    &= \frac{0.1}{2} (9.95 + 2 \times 4.9 + 2 \times 3.18 + 2 \times 2.3 + 1.76)\\
    &= \boxed{1.6235}
\end{align*}
\subsection*{روش سیمپسون}
کافی است که از فرمول داخل کتاب استفاده بکنیم:
\begin{align*}
    \int_{0.1}^{0.5} \frac{\cos x}{x} dx &= \frac{h}{3} (f_0 + 4f_1 + 2f_2 + 4f_3 + f_4)\\
    &= \frac{0.1}{3} (9.95 + 4 \times 4.9 + 2 \times 3.18 + 4 \times 2.3 + 1.76)\\
    &= \boxed{1.56233}
\end{align*}
\section*{سوال سوم}
\subsection*{روش ذوزنقه‌ای}
با توجه به فرمول داریم:
\begin{gather*}
    h = \sqrt{\frac{12 \times 0.001}{(1 - 0) |\max f''(x)|}}
\end{gather*}
حال باید
$|\max \operatorname{cosh}''(x^2)|$
زمانی که
$0 \le x \le 1$
است را پیدا بکنیم. از آنجا که اگر دو بار از
$\operatorname{cosh}(x)$
مشتق بگیریم همان
$\operatorname{cosh}(x)$
می‌شود پس در نتیجه ماکس آن در
$x = 1$
که نتیجه می‌دهد:
$|\max \operatorname{cosh}''(x^2)| \approx 8.523$
پس در نتیجه در کل داریم:
\begin{gather*}
    h = \sqrt{\frac{12 \times 0.001}{(1 - 0) 8.523}} \approx 0.0375\\
    n > \frac{1 - 0}{0.0375} = 26.66667\\
    \implies \boxed{n \ge 27}
\end{gather*}
\subsection*{روش سیمپسون}
داریم که خطا برابر است با:
\begin{gather*}
    \frac{b - a}{180}h^4 f^{(4)}(x)
\end{gather*}
در نتیجه مثل قسمت قبل داریم که
\begin{gather*}
    h = \sqrt[4]{\frac{180 \times 0.001}{(1 - 0) |f^{(4)}(x)|}}
\end{gather*}
دوباره به کمک
\lr{desmos}
داریم که حداکثر
$|f^{(4)}(x)|$
برابر با
$99.616$
است در نقطه‌ی 1. پس داریم:
\begin{gather*}
    h = \sqrt[4]{\frac{180 \times 0.001}{(1 - 0) 99.616}} \approx 0.206\\
    n > \frac{1 - 0}{0.206} = 4.8544\\
    \implies \boxed{n \ge 5}
\end{gather*}
\section*{سوال چهارم}
\subsection*{قسمت اول}
کافی است که از فرمول‌های سه نقطه‌دار استفاده بکنیم. به عنوان مثال در اینجا از فرمول
\lr{midpoint}
استفاده می‌کنیم.
\begin{align*}
    f'(4.5) &\approx \frac{1}{2h} (f(4.5 + 0.1) - f(4.5 - 0.1))\\
    &=\frac{1}{0.2} (-0.1087 - (-0.1136))\\
    &=\boxed{0.0245}
\end{align*}
\subsection*{قسمت دوم}
کافی است که از فرمول
\lr{forward}
استفاده بکنیم که بسیار شبیه تعریف خود مشتق است:
\begin{align*}
    f'(3) &\approx \frac{f(3 + 0.1) - f(3)}{h}\\
    &=\frac{-0.1612 - (-0.1666)}{0.1}\\
    &=\boxed{0.054}
\end{align*}
\section*{سوال پنجم}
در این قسمت باید از بسط تیلور استفاده بکنیم تا بتوانیم مشتق مرتبه سوم را حساب بکنیم.
\begin{align*}
    f(x_0 + h)  &= f(x_0) + \frac{f'(x_0)h}{1!} + \frac{f''(x_0)h^2}{2!} + \frac{f'''(x_0)h^3}{3!} + \frac{f^{(4)}(x_0)h^4}{4!} + O(h^5)\\
    f(x_0 + 2h) &= f(x_0) + \frac{f'(x_0)2h}{1!} + \frac{f''(x_0)4h^2}{2!} + \frac{f'''(x_0)8h^3}{3!} + \frac{f^{(4)}(x_0)16h^4}{4!} + O(h^5)\\
    f(x_0 + 3h) &= f(x_0) + \frac{f'(x_0)3h}{1!} + \frac{f''(x_0)9h^2}{2!} + \frac{f'''(x_0)27h^3}{3!} + \frac{f^{(4)}(x_0)81h^4}{4!} + O(h^5)\\
    f(x_0 + 4h) &= f(x_0) + \frac{f'(x_0)4h}{1!} + \frac{f''(x_0)16h^2}{2!} + \frac{f'''(x_0)64h^3}{3!} + \frac{f^{(4)}(x_0)256h^4}{4!} + O(h^5)\\
\end{align*}
حال کاری که باید انجام دهیم این است که
$f'''$
را برا اساس
$f(x_0 + ih)$ها
بنویسیم. یک نکته را توجه بکنید که اگر می‌خواستیم که خطا از اردر
$O(h)$
باشد باید عبارت‌هایی که
$h^4$
داشتند را حذف می‌کردیم چرا که در نهایت باید کل عبارت را تقسیم بر
$h^3$
می‌کردیم. و در نتیجه خطای عبارت از
$O(h)$
در می‌آمد. ولی در اینجا عبارت بعدی را نیز نوشته‌ایم که باعث می‌شود در نهایت خطا از
$O(h^2)$
شود. با ادامه دادن می‌توانستیم خطا‌‌های کمتری نیز برسیم. با حل کردن این معادله به عبارت زیر می‌رسیم:
\begin{gather*}
    f'''(x_0) = \frac{-3f(x_0 + 4h) + 14f(x_0 + 3h) - 24f(x_0 + 2h) + 18f(x_0 + 1h) - 5f(x_0)}{2h^3}
\end{gather*}
\section*{سوال ششم}
فرض کنید که چند جمله‌ای ما برابر است با
$P(x) = ax^3 + bx^2 + cx + d$. حال کافی است که صرفا در فرمول عددگذاری بکنیم.
\begin{align*}
    Q_f &= \frac{h}{2}(d + ah^3 + bh^2 + ch + d) - \frac{h^2}{12} (3ah^2 + 2bh)\\
    &= ah^4/4 + bh^3/2 + ch^2/2 + dh - ah^4/4 - bh^3/6\\
    &= \boxed{ah^4/4 + bh^3/3 + ch^2/2 + dh}
\end{align*}
همان طور که می‌دانیم انتگرال
$P(x)$
همان عبارت داخل باکس بالا است. پس برای چند جمله‌ای‌هایی تا درجه سه دقیق است.
برای قسمت دوم سوال کافی است که انتگرال را به قسمت‌های کوچکتر بشکنیم. یعنی عملا
\begin{align*}
    \int_{0}^{nh} f(x) dx &= \sum_{i=i}^{n} \int_{(i-1)h}^{ih} f(x)dx\\
    &= \sum_{i=i}^{n} \frac{h}{2}(f((i-1)h) + f(ih)) + \frac{h^2}{12} (f'((i-1)h) - f'(ih))\\
    &= \frac{h}{2}(f(0) + 2f(h) + 2f(2h) + \dots + 2f((n - 1)h) + f(nh)) + \frac{h^2}{12} (f'(0) - f'(nh))\\
    &= \frac{h}{2}(f(0) + 2 \bigl(\sum_{i=1}^{n-1} f(ih)\bigr) + f(h)) + \frac{h^2}{12} (f'(0) - f'(nh))
\end{align*}
همان طور که می‌بینید عبارت اول دقیقا همان انتگرال ذوزنقه‌ای در آمد.
\end{document}

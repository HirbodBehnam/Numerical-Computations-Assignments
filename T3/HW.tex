% !TEX program = xelatex
\documentclass[]{article}
\usepackage{commons/course}

\begin{document}
\printheader

\section*{سوال اول}

\section*{سوال دوم}
به صورت کلی از آنجا که
$n = 4$
است. پس نقاط ما برابر نقاط زیر هستند:
\begin{gather*}
    h = \frac{0.5 - 0.1}{4} = 0.1\\
    (0.1, 9.95), (0.2, 4.9), (0.3, 3.18), (0.4, 2.3), (0.5, 1.76)
\end{gather*}
\subsection*{روش ذوزنقه‌ای}
کافی است که از فرمول داخل کتاب استفاده بکنیم:
\begin{align*}
    \int_{0.1}^{0.5} \frac{\cos x}{x} dx &= \frac{h}{2} (f_0 + 2f_1 + 2f_2 + 2f_3 + f_4)\\
    &= \frac{0.1}{2} (9.95 + 2 \times 4.9 + 2 \times 3.18 + 2 \times 2.3 + 1.76)\\
    &= \boxed{1.6235}
\end{align*}
\subsection*{روش سیمپسون}
کافی است که از فرمول داخل کتاب استفاده بکنیم:
\begin{align*}
    \int_{0.1}^{0.5} \frac{\cos x}{x} dx &= \frac{h}{3} (f_0 + 4f_1 + 2f_2 + 4f_3 + f_4)\\
    &= \frac{0.1}{3} (9.95 + 4 \times 4.9 + 2 \times 3.18 + 4 \times 2.3 + 1.76)\\
    &= \boxed{1.56233}
\end{align*}
\section*{سوال سوم}
\subsection*{روش ذوزنقه‌ای}
با توجه به فرمول داریم:
\begin{gather*}
    n = \sqrt{\frac{12 \times 0.001}{(1 - 0) |\max f''(x)|}}
\end{gather*}
حال باید
$|\max \operatorname{cosh}''(x)|$
\subsection*{روش سیمپسون}
\section*{سوال چهارم}
\section*{سوال پنجم}
\section*{سوال ششم}
\end{document}

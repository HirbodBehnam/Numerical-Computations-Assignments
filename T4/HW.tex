% !TEX program = xelatex
\documentclass[]{article}
\usepackage{commons/course}

\begin{document}
\printheader

\section*{سوال اول}
\subsection*{قسمت اول}
در ابتدا باید متشق گیری بکنیم:
\begin{align*}
    y' &= x + y\\
    y'' &= 1 + y' = 1 + x + y\\
    y''' &= y'' = 1 + x + y\\
    y^{(4)} &= y''' = 1 + x + y
\end{align*}
حال چند بار بسط تیلور را اجرا می‌کنیم. به صورت کلی داریم:
\begin{align*}
    y_{n+1} &= y_n + \frac{0.1}{1!}(x_n + y_n) + \frac{(0.1)^2}{2!} (1 + x_n + y_n) + \frac{(0.1)^3}{3!} (1 + x_n + y_n) + \frac{(0.1)^4}{4!} (1 + x_n + y_n)\\
    &= 1.105170834 y + 0.1051708334 x + 0.005170833334
\end{align*}
پس داریم:
\begin{align*}
    y_0 &= 1\\
    y_1 &= 1.105170834 \times 1 + 0.1051708334 \times 0 + 0.005170833334 = 1.11034166733\\
    y_2 &= 1.105170834 \times 1.11034166733 + 0.1051708334 \times 0.1 + 0.005170833334 = 1.24280514318\\
    y_3 &= 1.105170834 \times 1.24280514318 + 0.1051708334 \times 0.2 + 0.005170833334 = 1.3997169966\\
    y_4 &= 1.105170834 \times 1.3997169966 + 0.1051708334 \times 0.3 + 0.005170833334 = 1.58364848385\\
    y_5 &= 1.105170834 \times 1.58364848385 + 0.1051708334 \times 0.4 + 0.005170833334 = \boxed{1.79744128235}\\
\end{align*}
\subsection*{قسمت دوم}
طبق کتاب داریم:
\begin{align*}
    w_{i+1} &= w_i + h\left(2\left(t_i + 1\right) \cos \left(2w_i\right)\right)\\
    w_0 &= 0\\
    h &= 0.1\\
    t_i &= 0.1i
\end{align*}
پس برای هر مرحله داریم:
\begin{align*}
    w_0 &= 0\\
    w_1 &= 0 + 0.1\left(2\left(0 + 1\right) \cos \left(2\times0\right)\right) = 0.2\\
    w_2 &= 0.2 + 0.1\left(2\left(0.1 + 1\right) \cos \left(2\times 0.2\right)\right) \approx 0.402\\
    w_3 &= 0.402 + 0.1\left(2\left(0.2 + 1\right) \cos \left(2\times 0.402\right)\right) \approx \boxed{0.568}
\end{align*}
\subsection*{قسمت سوم}
داریم:
\begin{align*}
    k_1 &= 0.1 f(1, 1) = 0.1 \times 1 \times 1^\frac{1}{3} = 0.1\\
    k_2 &= 0.1 f(1.05, 1.05) = 0.1 \times 1.05 \times 1.05^\frac{1}{3} \approx 0.106721617466\\
    k_3 &= 0.1 f(1.05, 1.053360808733) = 0.1 \times 1.05 \times 1.053360808733^\frac{1}{3} \approx 0.106835359989\\
    k_4 &= 0.1 f(1.1, 1.106835359989) = 0.1 \times 1.1 \times 1.106835359989^\frac{1}{3} = 0.113785527407\\
    w_1 &= 1 + (0.1 + 2 \times 0.106721617466 + 2 \times 0.106835359989 + 0.113785527407)
\end{align*}
حال به صورت مشابه و به کمک پایتون داریم:
\begin{latin}
    \centering
    \begin{tabular}{c|c|c}
        i & t & y\\
        \hline
        1 & 1.1 & 1.1067780165767351\\
        2 & 1.2 & 1.2277974969858367\\
        3 & 1.3 & 1.3640054127978642\\
    \end{tabular}
\end{latin}
\section*{سوال دوم}
\subsection*{قسمت اول}
برای این کار کافی است که صرفا
$y$
را در عبارت جایگزاری بکنیم:
\begin{align*}
    y' &= \frac{1 - x^2}{(1 + x^2)^2} \\
    &= \frac{1}{1 + x^2} - 2 \left(\frac{x}{1 + x^2}\right)^2\\
    &= \frac{1 + x^2 - 2x^2}{(1 + x^2)^2}\\
    &= \frac{1 - x^2}{(1 + x^2)^2}
\end{align*}
\subsection*{قسمت دوم}
\subsubsection*{اویلر}
\begin{align*}
    w_{i+1} &= w_i + h\left(\frac{1}{1+t_i^2}-2w_i^2\right)\\
    w_0 &= 0\\
    h &= 0.1\\
    t_i &= 0.1i
\end{align*}
پس برای هر مرحله داریم:
\begin{align*}
    w_0 &= 0\\
    w_1 &= 0 + 0.2\left(\frac{1}{1+0^2}-2 \times 0^2\right) = 0.2\\
    w_2 &= 0.2 + 0.2\left(\frac{1}{1+0.2^2}-2\times 0.2^2\right) \approx 0.376307692308\\
    w_3 &= 0.376307692308 + 0.2\left(\frac{1}{1+0.4^2}-2\times0.376307692308^2\right) \approx 0.492078493695\\
    w_4 &= 0.492078493695 + 0.2\left(\frac{1}{1+0.6^2}-2\times0.492078493695^2\right) \approx 0.542280819642\\
    w_5 &= 0.542280819642 + 0.2\left(\frac{1}{1+0.8^2}-2\times0.542280819642^2\right) \approx 0.546604644214\\
\end{align*}
\subsubsection*{هوین}
برای اولین نقطه داریم:
\begin{align*}
    k_1 &= 0.2 f(0, 0) = 0.2\left(\frac{1}{1+0^2}-2 \times 0^2\right) = 0.2\\
    k_2 &= 0.2 f(\frac{0.4}{3}, \frac{0.4}{3}) \approx 0.14317321688500728\\
    w_1 &= \frac{0.2 + 3 \times 0.14317321688500728}{4} \approx 0.15737991266375545
\end{align*}
به همین ترتیب به کمک پایتون داریم:
\begin{latin}
    \centering
    \begin{tabular}{c|c|c}
        i & t & y\\
        \hline
        0 & 0 & 0\\
        1 & 0.2 & 0.15737991266375545\\
        2 & 0.4 & 0.2516337252267383\\
        3 & 0.6 & 0.2965136297297697\\
        4 & 0.8 & 0.30652556199387004\\
        5 & 1 & 0.29470121391528936\\
    \end{tabular}
\end{latin}
\subsection*{قسمت سوم}
\begin{latin}
    \centering
    \begin{tabular}{c|c|c|c}
        $x$ & $y_{\text{actual}}$ & $|e_{\text{euler}}|$ & $|e_{\text{heun}}|$\\
        \hline
        0 & 0 & 0 & 0\\
        0.2 & 0.192307692308 & 0.00769230769231 & 0.0349277796439\\
        0.4 & 0.344827586207 & 0.0314801061011 & 0.0931938609802 \\
        0.6 & 0.441176470588 & 0.0509020231068 & 0.144662840858\\
        0.8 & 0.487804878049 & 0.0544759415932 & 0.181279316055\\
        1 & 0.5 & 0.046604644214 & 0.205298786085\\
    \end{tabular}
\end{latin}
\section*{سوال چهارم}
برای این سوال فرض می‌کنیم که
$h = 2$
است. سپس با این حال در ابتدا برای پیدا کردن یک مقدار اولیه دیگر از روش اویلر استفاده می‌کنیم:
\begin{gather*}
    y(2) \approx y(0) + h f(0, y(0)) = 6 + 0.5 (0.2 \times 6 - 0.1 \times 6) = 1.2
\end{gather*}
حال در ادامه داریم:
\begin{align*}
    y_2 &= y_1 + \frac{3}{2} \times 2 (0.2 \times 1.2000 - 0.1 \times 1.2000) - \frac{2}{2} (0.2 \times 6.0000 - 0.1 \times 6.0000) = 1.2\\
y_3 &= y_2 + \frac{3}{2} \times 2 (0.2 \times 3.8880 - 0.1 \times 3.8880) - \frac{2}{2} (0.2 \times 1.2000 - 0.1 \times 1.2000) = 3.888\\
y_4 &= y_3 + \frac{3}{2} \times 2 (0.2 \times 1.5898 - 0.1 \times 1.5898) - \frac{2}{2} (0.2 \times 3.8880 - 0.1 \times 3.8880) = 1.59\\
y_5 &= y_4 + \frac{3}{2} \times 2 (0.2 \times 2.5195 - 0.1 \times 2.5195) - \frac{2}{2} (0.2 \times 1.5898 - 0.1 \times 1.5898) = 2.52\\
y_6 &= y_5 + \frac{3}{2} \times 2 (0.2 \times 2.0616 - 0.1 \times 2.0616) - \frac{2}{2} (0.2 \times 2.5195 - 0.1 \times 2.5195) = 2.062\\
y_7 &= y_6 + \frac{3}{2} \times 2 (0.2 \times 2.1544 - 0.1 \times 2.1544) - \frac{2}{2} (0.2 \times 2.0616 - 0.1 \times 2.0616) = 2.154\\
y_8 &= y_7 + \frac{3}{2} \times 2 (0.2 \times 2.0673 - 0.1 \times 2.0673) - \frac{2}{2} (0.2 \times 2.1544 - 0.1 \times 2.1544) = 2.067\\
y_9 &= y_8 + \frac{3}{2} \times 2 (0.2 \times 2.0588 - 0.1 \times 2.0588) - \frac{2}{2} (0.2 \times 2.0673 - 0.1 \times 2.0673) = 2.059\\
y_{10} &= y_9 + \frac{3}{2} \times 2 (0.2 \times 2.0364 - 0.1 \times 2.0364) - \frac{2}{2} (0.2 \times 2.0588 - 0.1 \times 2.0588) = \boxed{2.036}
\end{align*}
\end{document}

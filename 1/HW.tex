% !TEX program = xelatex
\documentclass[]{article}
\usepackage{commons/course}

\begin{document}
\printheader

\section*{سوال اول}
به صورت کلی می‌دانیم که چند جمله‌ای درجه
$n$
تیلور حول نقطه 0 برابر است با
\begin{gather*}
    \sum_{i=0}^{n} \frac{f^{(i)}(0)}{i!}(x)^i
\end{gather*}
حال با توجه به این موضوع قسمت‌های این سوال را حل می‌کنیم:
\subsection*{قسمت اول}
در ابتدا مشخص است که این عبارت برابر است با $\ln(1 + x) - \ln(1 - x)$. پس بسط تیلور را برای دو $\ln$ می‌نویسیم. برای شروع مشخصا از مشتق‌ها شروع می‌کنیم:
\begin{align*}
    f(x) &= \ln(1 + x)\\
    f'(x) &= \frac{1}{1 + x} = (1 + x)^{-1}\\
    f''(x) &= -(1 + x)^{-2}\\
    f'''(x) &= 2(1 + x)^{-3}\\
    f''''(x) &= -6(1 + x)^{-4}\\
    &\vdots\\
    f^{(n)}(x) &= (-1)^{(n-1)} (n-1)! (1+x)^{-n} \quad x > 0
\end{align*}
مشخص است که
$f^{(n)}(0) = (-1)^{(n-1)} (n-1)!$
است. از آن طرف به صورت مشابه داریم:
\begin{align*}
    g(x) &= \ln(1 - x)\\
    g'(x) &= -\frac{1}{1 - x} = -(1 - x)^{-1}\\
    g''(x) &= (1 - x)^{-2}\\
    g'''(x) &= -2(1 - x)^{-3}\\
    g''''(x) &= 6(1 - x)^{-4}\\
    &\vdots\\
    g^{(n)}(x) &= (-1)^{n} (n-1)! (1-x)^{-n} \quad x > 0
\end{align*}
مشخص است که
$g^{(n)}(0) = (-1)^{n} (n-1)!$
است. همچنین دقت کنید که جمله‌ی صفرم در هر دو عبارت برابر 0 است
است چرا که $\ln 1 = 0$.
پس در نهایت داریم برای جمله‌ی
$n$ام
تیلور این عبارت که:
\begin{align*}
    &\sum_{i=1}^{n} \frac{(-1)^{(i-1)} (i-1)!}{i!}(x)^i - \frac{(-1)^{i} (i-1)!}{i!}(x)^i\\
    =&\sum_{i=1}^{n} \frac{(-1)^{(i-1)}}{i}(x)^i - \frac{(-1)^{i}}{i}(x)^i\\
    =&\sum_{i=1}^{n} \frac{(-1)^{(i-1)}}{i}(x)^i + \frac{(-1)^{(i-1)}}{i}(x)^i\\
    =&\boxed{2\sum_{i=1}^{n} \frac{(-1)^{(i-1)}}{i}(x)^i}
\end{align*}
\subsection*{قسمت دوم}
با توجه به \link{https://en.wikipedia.org/wiki/Inverse_hyperbolic_functions\#Derivatives}{ویکیپدیا}:
\begin{gather*}
    \frac{d}{dx}\operatorname{arcsinh}(x) = \frac{1}{\sqrt{(x^2+1)}}\\
    \operatorname{arcsinh}(0) = 0
\end{gather*}
پس عملا داریم:
\begin{align*}
    f(0) &= \operatorname{arcsinh}(0) = 0\\
    f'(0) &= \frac{1}{\sqrt{(x^2+1)}} = (x^2+1)^{-\frac{1}{2}} = (0^2+1)^{-\frac{1}{2}} = 1\\
    f''(0) &= -\frac{2x}{2} (x^2+1)^{-\frac{3}{2}} = 0\\
    f'''(0) &= 3x^2 (x^2+1)^{-\frac{5}{2}} - (x^2+1)^{-\frac{3}{2}} = -1\\
    f''''(0) &= 6x (x^2+1)^{-\frac{5}{2}} - 15x^2 (x^2+1)^{-\frac{7}{2}} + 3x^2 (x^2+1)^{-\frac{5}{2}} \\
    &= 9x (x^2+1)^{-\frac{5}{2}} - 15x^2 (x^2+1)^{-\frac{7}{2}} = 0\\
    f'''''(0) &= 9 (x^2+1)^{-\frac{5}{2}} = 9
\end{align*}
حال پس برای بسط تیلور داریم:
\begin{align*}
    &\sum_{n=0}^{5} \frac{f^{(n)}(0)}{n!}(x)^n\\
    =&\frac{1}{1!}(x)^1 + \frac{-1}{3!}(x)^3 + \frac{9}{5!}(x)^5\\
    =&\boxed{x - \frac{x^3}{6} + \frac{3x^5}{40}}
\end{align*}
\subsection*{قسمت سوم}
TODO
\section*{سوال دوم}
\subsection*{قسمت اول}
در ابتدا داریم:
\begin{align*}
    f(0) &= (1 + x)^{-1} = 1\\
    f'(0) &= -(1 + x)^{-2} = -1\\
    f''(0) &= 2(1 + x)^{-3} = 2\\
    f'''(0) &= -6(1 + x)^{-4} = -6\\
    &\vdots\\
    f^{(n)}(0) &= (-1)^{n} (n)!
\end{align*}
پس بسط ما برابر است با:
\begin{gather*}
    \sum_{n=0}^{\infty} \frac{(-1)^{n} (n)!}{n!}(x)^n = (-x)^n
\end{gather*}
حال عملا اگر جای
$x$ قرار دهیم $e^{-x}$
داریم:
\begin{gather*}
    \frac{1}{1+e^{-x}} = \boxed{\sum_{n=0}^{\infty} (-e^{-x})^n}
\end{gather*}
می‌دانیم که
$e^x = \sum_{n=0}^{\infty} \frac{1}{n!}x^n$
است. پس
$e^{-x} = \sum_{n=0}^{\infty} \frac{1}{n!}(-x)^n$
است. پس داریم:
\begin{gather*}
    \frac{1}{1+e^{-x}} = \sum_{n=0}^{\infty} (- \sum_{m=0}^{\infty} \frac{1}{m!}(-x)^m)^n
\end{gather*}
اما چه طور می‌توان این عبارت را ساده‌تر کرد؟ به نظر می‌رسد که نمی‌شود! من کمی در اینترنت سرچ کردم
و به عنوان مثال به
\link{https://math.stackexchange.com/a/1404965}{این} 
جواب رسیدم که تقریبا شبیه راه حل مد نظر سوال است. اما در بالاتر از آن نیز یک راه بسیار زیبای دیگر وجود
دارد که در
\link{https://math.stackexchange.com/a/1405004}{اینجا}
قابل مشاهده است و صرفا بازی کردن با ضریب‌ها است. اما اگر بخواهیم که با راه مد نظر طراح جلو برویم
باید ببینیم که در چه حالتی جمله‌ی ما
$x^0$
تا
$x^n$
است و ضریب‌های آن‌ها را با هم جمع کنیم. اما این کار در اینجا ممکن نیست چرا که اگر به عنوان مثال
جمله‌های بدون
$x$
نگاه کنید متوجه می‌شوید که بین -1 و 1 همین طوری در حال چرخش هست. پس در نهایت از راهی شبیه راه
\lr{math stackexchange}
استفاده می‌کنیم:
\begin{align*}\\
    f(x)
    &=\frac{1}{1+e^x}\\
    &=\frac{1}{2+(e^x-1)}\\
    &=\frac12\frac{1}{1+(e^x-1)/2}\\
    &=\frac12\sum_{n=0}^{\infty}(\frac{1-e^x}{2})^n\\
    &=\frac12\sum_{n=0}^{\infty}(\frac{1-(1+x+x^2/2+x^3/6+\dots)}{2})^n\\
    &=\frac12\sum_{n=0}^{\infty}(-\frac{x+x^2/2+x^3/6+\dots}{2})^n\\
    &=\frac12\sum_{n=0}^{\infty}(-\frac{x}{2} (1+x/2+x^2/6+\dots))^n\\
    &=\frac12\sum_{n=0}^{\infty}(-1)^n(x/2)^n(1+x/2+x^2/6+\dots)^n\\
    &=\frac12\sum_{n=0}^{\infty}(-1)^n(x/2)^n(1+nx/2+\dots)\\
    &=\frac12\left(1-\frac{x}{2}(1+\frac{x}{2}+\dots)+\frac{x^2}{4}(1+\dots)+\dots\right)\\
    &=\frac12\left(1-\frac{x}{2}-\frac{x^2}{4} +\frac{x^2}{4}(1+\dots)+\dots\right)\\
    &=\frac12\left(1-\frac{x}{2}+O(x^3)\right)\\
    &=\frac12-\frac{x}{4}+O(x^3)\\
\end{align*}
به همین ترتیب می‌توان جلو رفت و به عنوان مثال برای 5 جمله‌ی اول داریم:
\begin{gather*}
    \boxed{\frac12+\frac{1}{4}x-\frac{1}{48}x^3 + \frac{1}{480}x^5 - \frac{17}{80640}x^7}
\end{gather*}
\subsection*{قسمت دوم}
از باز شده‌ی این عبارت استفاده نمی‌کنیم. بلکه از
$\frac{1}{1+e^{-x}} = \sum_{n=0}^{\infty} (-e^{-x})^n$
استفاده می‌کنیم. دقت کنید که با ضرب کردن دو طرف تساوی در
$e^{-x}$
و کمی مرتب سازی داریم:
\begin{gather*}
    \frac{e^{-x}}{1+e^{-x}} = \sum_{n=0}^{\infty} (-1)^n {e^{-x}}^{n+1} = \sum_{n=0}^{\infty} (-1)^n {e^{-(n+1)x}}
\end{gather*}
حال با جایگزای عبارت بالا در انتگرال داریم:
\begin{align*}
    \int_{0}^{\infty} \frac{e^{-x}}{1+e^{-x}} dx &= \int_{0}^{\infty} \sum_{n=0}^{\infty} (-1)^n {e^{-(n+1)x}}\\
    &= \int_{0}^{\infty} e^{-x} - e^{-2x} + e^{-3x} - e^{-4x} + \dots dx\\
    &= \int_{0}^{\infty} e^{-x} dx - \int_{0}^{\infty} e^{-2x} dx + \int_{0}^{\infty} e^{-3x} dx - \int_{0}^{\infty} e^{-4x} dx + \dots\\
    &= 1 - \frac{1}{2} + \frac{1}{3} - \frac{1}{4} + \dots\\
    &= \sum_{n=1}^{\infty} \frac{(-1)^{n+1}}{n}
\end{align*}
\subsection*{قسمت سوم}
به کمک تغییر متغیر داریم
$e^{-x} = u$
و در نتیجه
$-e^{-x} dx = du$
و در نهایت داریم:
\begin{align*}
    \int_{0}^{\infty} \frac{e^{-x}}{1+e^{-x}} dx &= \int_{0}^{\infty} \frac{-1}{1+u} du\\
    &= -\Big[\ln 1 + u\Big]_{0}^{\infty}\\
    &= -\Big[\ln 1 + e^{-x}\Big]_{0}^{\infty}\\
    &= \ln (1 + 1) - \ln 1\\
    &= \boxed{\ln 2}
\end{align*}
\section*{سوال سوم}
\section*{سوال چهارم}
می‌دانیم که مقدار خطای محاسبه اگر که تا جمله‌ی
\lr{n}ام
بسط را بنویسیم جایی بین
$\frac{\cos^{(n+1)}(0)}{(n+1)!}(\pi / 9)^{n+1}$ و $\frac{\cos^{(n+1)}(\pi / 9)}{(n+1)!}(\pi / 9)^{n+1}$
است. حال با این حال شروع به حساب کردن بسط تیلور می‌کنیم تا جایی که تحت هیچ حالتی خطا بیشتر از
$0.001$
نباشد. همچنین دقت کنید که ما باید قدر مطلق خطا را در نظر بگیریم در اینجا.
\begin{align*}
    n \coloneqq 0 &\implies -\frac{\sin(\pi / 9)}{1!}(\pi / 9)^{1} \leq E \leq -\frac{\sin(0)}{1!}(\pi / 9)^{1}\\
    n \coloneqq 1 &\implies -\frac{\cos(\pi / 9)}{2!}(\pi / 9)^2 \geq E \geq -\frac{\cos(0)}{2!}(\pi / 9)^{2} \approx -0.06\\
    n \coloneqq 2 &\implies \frac{\sin(0)}{3!}(\pi / 9)^3 \leq E \leq \frac{\sin(\pi / 9)}{3!}(\pi / 9)^{3}\\
    n \coloneqq 3 &\implies \frac{\cos(\pi / 9)}{4!}(\pi / 9)^{4} \leq E \leq \frac{\cos(0)}{4!}(\pi / 9)^4 \approx 0.0006
\end{align*}
پس در اینجا در صورتی که تا جمله‌ی
$n=2$
بنویسم کافی است. داریم:
\begin{align*}
    \cos(\pi/9) &\approx \frac{\cos(0)}{0!}(\pi / 9)^{0} - \frac{\sin(0)}{1!}(\pi / 9)^{1} - \frac{\cos(0)}{2!}(\pi / 9)^{2}\\
    &= 1 - 0 - \frac{1}{2}(\pi / 9)^{2}\\
    &= 0.939076516043
\end{align*}
همچنین در نهایت در صورتی که به کمک ماشین حساب مقدار
$\cos(\pi/9)$
را حساب کنیم می‌بینیم که برابر
$0.939692620786$
است. این نشان می‌دهد که مقدار خطا برابر
$0.000616104743342$
است. همچنین در نهایت لازم است که دو نکته را مشخص کنم. در ابتدا دقت کنید که در بازه‌ی
$0$ تا $\frac{\pi}{9}$
تابع کسینوس نزولی و سینوس صعودی است. پس در نتیجه برای حساب کردن
\lr{upperbound}
خطا هر جا که دو طرف خطا سینوس است باید
\lr{upperbound}
را به وسیله طرفی که
$\sin(\pi / 9)$
دارد مشخص کنیم. اما مشکل اینجا است که مقدار این عبارت را نداریم! به همین منظور مجبور هستیم که
از عباراتی استفاده کنیم که در آنها کسینوس ظاهر شده است چرا که کسینوس ۰ را داریم.
\section*{سوال پنجم}
\subsection*{قسمت اول}
در ابتدا یک سری مشتق را حساب می‌کنیم:
\begin{align*}
    f(1,1,1) &= \frac{e^{2}}{2}\\
    \frac{\partial}{\partial x} f(1,1,1) &= \frac{2xe^{x^2+yz}}{y + z} = \frac{2e^{2}}{2} = e^x\\
    \frac{\partial}{\partial y} f(1,1,1) &= \frac{ze^{x^2+yz}}{y + z} - \frac{e^{x^2+yz}}{(y + z)^2} = \frac{e^{2}}{2} - \frac{e^{2}}{4} = \frac{e^{2}}{4}\\
    \frac{\partial}{\partial z} f(1,1,1) &= \frac{ye^{x^2+yz}}{y + z} - \frac{e^{x^2+yz}}{(y + z)^2} = \frac{e^{2}}{4}\\
    \frac{\partial}{\partial^2 x} f(1,1,1) &= \frac{2e^{x^2+yz}+4x^2e^{x^2+yz}}{y + z} = \frac{6e^2}{2} = 3e^x\\
    \frac{\partial}{\partial^2 y} f(1,1,1) &= \frac{z^{2} e^{x^{2}+y z}}{y +z}-\frac{2 z e^{x^{2}+y z}}{\left(y +z \right)^{2}}+\frac{2 e^{x^{2}+y z}}{\left(y +z \right)^{3}} = \frac{e^{2}}{4}\\
    \frac{\partial}{\partial^2 z} f(1,1,1) &= \frac{y^{2} e^{x^{2}+y z}}{y +z}-\frac{2 y e^{x^{2}+y z}}{\left(y +z \right)^{2}}+\frac{2 e^{x^{2}+y z}}{\left(y +z \right)^{3}} = \frac{e^{2}}{4}\\
    \frac{\partial}{\partial x \partial y} f(1,1,1) &= \frac{2 z x e^{x^{2}+y z}}{y +z}-\frac{2 x e^{x^{2}+y z}}{\left(y +z \right)^{2}} = \frac{e^{2}}{2}\\
    \frac{\partial}{\partial x \partial z} f(1,1,1) &= \frac{2 x y e^{x^{2}+y z}}{y +z}-\frac{2 x e^{x^{2}+y z}}{\left(y +z \right)^{2}} = \frac{e^{2}}{2}\\
    \frac{\partial}{\partial y \partial z} f(1,1,1) &= \frac{e^{x^{2}+y z}}{y +z}+\frac{z y e^{x^{2}+y z}}{y +z}-\frac{z e^{x^{2}+y z}}{\left(y +z \right)^{2}}-\frac{y e^{x^{2}+y z}}{\left(y +z \right)^{2}}+\frac{2 e^{x^{2}+y z}}{\left(y +z \right)^{3}} = \frac{3e^{2}}{4}\\
\end{align*}
حال خود بسط تیلور را می‌نویسیم:
\begin{align*}{2}
    \frac{e^{x^2+yz}}{y + z} &\approx \frac{e^{2}}{2} &&\\
    &+ (x - 1) e^x + (y - 1) \frac{e^{2}}{4} + (z - 1) \frac{e^{2}}{4}&&\\
    &+ \frac{1}{2} \biggl(\left(x - 1\right)^2 3e^x + \left(y - 1\right)^2 \frac{e^{2}}{4} + \left(z - 1\right)^2 \frac{e^{2}}{4} \\
    &+ 2\left(x - 1\right)\left(y - 1\right) \frac{e^{2}}{2} + 2\left(x - 1\right)\left(z - 1\right) \frac{e^{2}}{2} + 2\left(y - 1\right)\left(z - 1\right) \frac{3e^{2}}{4}\biggr)\\
\end{align*}
\subsection*{قسمت دوم}
\end{document}
